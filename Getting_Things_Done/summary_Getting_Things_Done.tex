\documentclass[letterpaper]{article}
\usepackage[utf8]{inputenc}				% use Unicode
\setlength\parindent{0pt}				% no indentation at start of paragraph
\usepackage{graphicx}					% for figures

\newcommand{\p}{\vspace{1em}\par}		% macro to start new paragraph

\author{David De Lille}
\title{Summary: Getting Things Done}

\begin{document}
\maketitle

\section{Chapter 1: A New Practice for a New Reality}
Goal: function productively with a clear head and a positive sense of relaxed control, even while having an overwhelming number of things to do.

\p Key objectives on which the methods are based:
\begin{enumerate}
\item capturing \textbf{all} the things that need to get done into a logical and trusted system outside of your head and off your mind
\item making front-end decisions about all your \textit{stuff} so that you will always have a plan for ``next actions'' that you can implement/renegotiate at any moment
\end{enumerate}

\subsection{The Problem: New Demands, Insufficient Resources}
Compared to the past, people have enhanced quality of life, but they are taking on more than they have resources to handle (adding to their stress levels).

\subsubsection*{Work No Longer Has Clear Boundaries}
Work used to be self-evident. Now, there are no edges to most of our projects. Even if we had the rest of their lives to try, we wouldn't be able to finish them to perfection. 

\subsubsection*{Our Jobs Keep Changing}
The definition of our jobs constantly shifting, because:
\begin{itemize}
\item organizations are in constant morph mode (constantly adapting to changing markets, clients, etc.)
\item the average professional is more of a free agent than ever before
\end{itemize}

\subsubsection*{The Old Models and Habits Are Insufficient}
Existing processes or tools are unable to accommodate the speed, complexity, and changing priority factors of our work.

\subsubsection*{The ``Big Picture'' vs. the Nitty-Gritty}
The ``bigger view'' has been presented as the solution to dealing with our complex world. Clarifying major goals and values should give order, meaning, and direction to our work. In practice, however, values thinking does not achieve its desired results, for the following reasons:
\begin{itemize}
\item too many day-to-day or hour-to-hour distractions to focus on the higher levels
\item ineffective personal organizational systems create subconscious resistance to undertaking even bigger projects, and cause more distraction and stress
\item clarifying higher levels and values raises the bar of our standards, making us notice that much more that needs changing (adding to the overwhelming number of things we have to do)
\end{itemize}
Focusing on primary outcomes and values is a critical exercise, certainly. But, it does not make things simpler. 

\p We need a system with a coherent set of behaviors and tools that functions effectively at the level at which work really happens. It must:
\begin{itemize}
\item incorporate the results of big-picture thinking as well as the smallest of open details
\item manage multiple tiers of priorities
\item maintain control over hundreds of new inputs daily
\item save a lot more time and effort than are needed to maintain it
\item make it easier to get things done
\end{itemize}

\subsection{The Promise: The ``Ready State'' of the Martial Artist}
It is possible to have your personal management situation totally under control, and to dedicate 100\% of your attention to whatever it at hand, at your own choosing, with no distraction. This is called a ``mind like water''.

\subsubsection*{The ``Mind Like Water'' Simile}
Imagine throwing a pebble into a still pond. The water responds totally appropriately to the force and mass of the input; then it returns to calm. It doesn't overreact or underreact. Anything that causes you to overreact or underreact can control you, and often does.

\subsubsection*{Can You Get into Your ``Productive State'' When Required?}
The methodology of this book will show you how to get into a state that:
\begin{itemize}
\item gives a sense of being in control
\item doesn't stress you out
\item keeps you highly focused (and makes time tend to disappear)
\item makes you feel like you are making noticeable progress
\end{itemize}

\subsection{The Principle: Dealing Effectively with Internal Commitments}
Most stress comes from inappropriately managed commitments. Even people who are not consciously ``stressed out'' will be more relaxed, have better focus, and have increased productive energy when they learn to more effectively control the ``open loops'' of their lives.

\p All the agreements with yourself (big or little) are being tracked by a less-than-conscious part of you. These are the ``incompletes'', or ``open loops'': anything pulling at your attention that doesn't belong where it is, the way it is. It's likely that you also have more internal commitments currently in play than you're aware of. You must first identify and collect all those things that are ``ringing your bell'' in some way, and then plan how to handle them.

\subsubsection*{The Basic Requirements for Managing Commitments}
\begin{itemize}
\item capturing anything you consider unfinished in a trusted system (collection bucket) outside of your mind, that you'll come back to and sort regularly
\item clarifying the commitment and deciding what you need to do to make progress on it
\item keep reminders of the actions you need to take, in a system you review regularly
\end{itemize}

\subsubsection*{An Important Exercise to Test This Model}
Write down the project or situation that is most on your mind at this moment. Describe, in a single written sentence, your intended successful outcome for this problem or situation (examples: ``Take the Hawaii vacation'', ``Handle situation with customer X'', ``Resolve college situation with Susan'').

\p Write down the very next physical action required to move the situation forward. If you had nothing else to do in your life but get closure on this, where would you go right now, and what visible action would you take?

\p After this small exercise, you'll probably be experiencing at least a tiny bit of enhanced control, relaxation, and focus. What happened is that you acquired a clearer definition of the outcome desired and the next action required.

\subsubsection*{Why Things Are on Your Mind}
The reason something is on your mind is that you want it to be different than it currently is, but you haven't clarified exactly what the intended outcome is, decided what the very next physical action step is, and/or put reminders of the outcome and the action required in a system you trust. Your brain can't give up the job until this has been done.

\subsubsection*{Your Mind Doesn't Have a Mind of Its Own}
At least a portion of your mind is really kind of stupid; it keeps reminding you about things when you can't do anything about them. It's a waste of time and energy to keep thinking about something that you make no progress on.

\subsubsection*{The Transformation of ``Stuff''}
``Stuff'' = anything you have allowed into your psychological or physical world that doesn't belong where it is, but for which you haven't yet determined the desired outcome and the next action step. As long as it's still ``stuff'', it's not controllable. ``Stuff'' is not inherently a bad thing. But once ``stuff'' comes into our lives and work, we should define and clarify its meaning.

\subsection{The Process: Managing Action}
You can train yourself to:
\begin{itemize}
\item be faster, more responsive, proactive, and focused in knowledge work
\item think more effectively and manage the results with more ease and control
\item minimize the loose ends across the whole spectrum of your life and get more done with less effort
\item make front-end decision-making about all the ``stuff'' you collect and create standard operating procedure for working more effectively
\end{itemize}
But first, you'll need to get in the habit of keeping nothing on your mind. The key to managing all of your ``stuff'' is managing your actions (and not managing time, information, or priorities).

\subsubsection*{Managing Action Is the Prime Challenge}
It's extremely difficult to manage actions you haven't identified or decided on. Projects can seem overwhelming, because you can't do a project; you can only do an action related to it.

\p The real problem for a lot of people is not a lack of time, but a lack of clarity and definition about what a project really is, and what the associated next-action steps required are.

\subsection*{The Value of a Bottom-Up Approach}
Intellectually, the most appropriate way ought to be to work from the top down:
\begin{enumerate}
\item uncovering personal and corporate missions
\item defining critical objectives
\item focusing on the details of implementation
\end{enumerate}
However, most people are so embroiled in commitments on a day-to-day level that they can't focus on the larger horizon. Consequently, a bottom-up approach is usually more effective.

\p You'll be better equipped to undertake higher-focused thinking when your tools for handling the resulting actions for implementation are part of your ongoing operational style. There are more meaningful things to think about than your in-basket, but if your management of that level is not as efficient as it could be, it's like trying to swim in baggy clothing.

\subsubsection*{Horizontal and Vertical Action Management}
You need to control commitments, projects, and actions in two ways: horizontally and vertically. ``Horizontal'' control maintains coherence across all the activities in which you are involved. ``Vertical'' control manages thinking up and down the track of individual topics and projects. The goal for managing horizontally and vertically is the same: to get things off your mind and get things done.

\subsubsection*{The Major Change: Getting It All Out of Your Head}
The individual behaviors described in this book are things you're already doing. The big difference between what I do and what others do, is that I capture and organize 100\% of my ``stuff'' in and with objective tools at hand, not in my mind. Everything: little or big, personal or professional.

\p I try to make intuitive choices based on my options, instead of trying to think about what those options are. I need to have thought about all of that already and captured the results in a trusted way. I don't want to waste time thinking about things more than once.

\p The short-term-memory part of your mind (the part that holds all of the incomplete, undecided, and unorganized ``stuff'') functions much like RAM, which has a limited capacity; there's only so much ``stuff'' you can store in there and still function at a high level. As soon as you have two things to do stored in your RAM, you've generated personal failure, because you can't do them both at the same time.

\section{Chapter 2: Getting Control of Your Life: The Five Stages of Mastering Workflow}
Five-stage method for managing workflow (horizontally):
\begin{enumerate}
\item \textbf{collect} things that command our attention
\item \textbf{process} what they mean and what to do about them
\item \textbf{organize} the results
\item \textbf{review} as options
\item \textbf{do}
\end{enumerate}
The method is straightforward, but most people can significantly improve their handling of these stages. The quality of your workflow management is only as good as the weakest link in this five-phase chain. If any one of these links is weak, what someone is likely to choose to do at any point in time may not be the best option. 

\p It is helpful to separate these stages as throughout the day. The reason people have trouble getting organized, is that they try to do all five phases at once.

\subsection{Collect}
\subsubsection*{Gathering 100\% of the ``Incompletes''}
You need to collect and gather together placeholders for or representations of all the things you consider incomplete in your world; anything that you think ought to be different than it currently is. You need to capture these open loops into ``containers'' that hold items until you decide what they are and what you're going to do about them.

\subsubsection*{The Collection Success Factors}
Three requirements to make the collection phase work:
\begin{itemize}
\item every open loop must be in your collection system and out of your head
\item you must have as few collection buckets as you can get by with
\item you must empty them regularly
\end{itemize}
This collection system should become part of your life-style. You need this system to be available to you in every context, since things you'll want to capture may show up almost anywhere.

\subsection{Process}
p31

\end{document}
% find "
% replace "stuff" by ``stuff''