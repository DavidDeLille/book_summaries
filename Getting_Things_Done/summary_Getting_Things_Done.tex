\documentclass[letterpaper]{article}
\usepackage[utf8]{inputenc}				% use Unicode
\setlength\parindent{0pt}				% no indentation at start of paragraph
\usepackage{graphicx}					% for figures

\newcommand{\p}{\vspace{1em}\par}		% macro to start new paragraph

\author{David De Lille}
\title{Summary: Getting Things Done}

\begin{document}
\maketitle

\section{Chapter 1: A New Practice for a New Reality}
Goal: function productively with a clear head and a
positive sense of relaxed control, while having an overwhelming number of
things to do.

\p Key objectives on which the methods are based:
\begin{enumerate}
\item capturing \textbf{all} the things that need to get done into a logical and trusted system outside of your head and off your mind
\item make front-end decisions about all your \textit{stuff} so that you will always have a plan for "next actions" that you can implement/renegotiate at any moment
\end{enumerate}

\subsection{The Problem: New Demands, Insufficient Resources}
A paradox has emerged in this new millennium: people have enhanced quality of life, but they are taking on more than they have resources to handle (adding to their stress levels).

\subsubsection*{Work No Longer Has Clear Boundaries}
In the old days, work was self-evident. Now, there are no edges to most of our projects (even if we had the rest of their lives to try, we wouldn't be able to finish these to perfection). 

\subsubsection*{Our Jobs Keep Changing}
The definition of our jobs constantly shifting, because:
\begin{enumerate}
\item organizations are in constant morph mode (adapting to changing markets, clients, etc.)
\item average professional is more of a free agent than ever before
\end{enumerate}

\subsubsection*{The Old Models and Habits Are Insufficient}
Nothing (including education, time-management models, or existing organizing tools) has provided us with a viable means of meeting the new demands placed on us. These processes or tools are unable to accommodate the speed, complexity, and changing priority factors of what we are doing.

\subsubsection*{The ``Big Picture'' vs. the Nitty-Gritty}
A huge number of business books, models, seminars, and gurus have championed the ``bigger view'' as the solution to dealing with our complex world. Clarifying major goals and values should give order, meaning, and direction to our work. In practice, however, the well-intentioned exercise of values thinking too often does not achieve its desired results, for the following reasons:
\begin{enumerate}
\item too many day-to-day/hour-to-hour distractions to focus on the higher levels
\item ineffective personal organizational systems create huge subconscious resistance to undertaking even bigger projects/goals, and cause even more distraction and stress
\item clarifying higher levels and values raises the bar of our standards, making us notice that much more that needs changing (adding to the overwhelming number of things we have to do)
\end{enumerate}
Focusing on primary outcomes and values is a critical exercise, certainly. But, it does not make things simpler. 

\p We need a system with a coherent set of behaviors and tools that functions effectively at the level at which work really happens. It must:
\begin{itemize}
\item incorporate the results of big-picture thinking as well as the smallest of open details
\item manage multiple tiers of priorities
\item maintain control over hundreds of new inputs daily
\item save a lot more time and effort than are needed to maintain it
\item make it easier to get things done
\end{itemize}

\subsection{The Promise: The ``Ready State'' of the Martial Artist}
It is possible to have your personal management situation totally under control, and to dedicate 100\% of your attention to whatever it at hand, at your own choosing, with no distraction. This is what martial artists call a `mind like water'' (top athletes call it the ``zone''). It's a condition of working, doing, and being in which the mind is clear and constructive things are happening.

\subsubsection*{The ``Mind Like Water'' Simile}
Imagine throwing a pebble into a still pond. How does the water respond? Totally appropriately to the force and mass of the input; then it returns to calm. It doesn't overreact or underreact. Anything that causes you to overreact or underreact can control you, and often does.

\subsubsection*{Can You Get into Your ``Productive State'' When Required?}
The methodology of this book will show you how to into a state that:
\begin{itemize}
\item gives a sense of being in control
\item doesn't stress you out
\item keeps you highly focused (and makes time tend to disappear)
\item makes you feel like you are making noticeable progress
\end{itemize}

\subsection{The Principle: Dealing Effectively with Internal Commitments}
Most of the stress people experience comes from inappropriately managed commitments they make or accept. Even those who are not consciously ``stressed out'' will invariably experience greater relaxation, better focus, and increased productive energy when they learn more effectively to control the ``open loops'' of their lives.





\end{document}

